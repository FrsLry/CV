%%%%%%%%%%%%%%%%%%%%%%%%%%%%%%%%%%%%%%%%%
% "ModernCV" CV and Cover Letter
% LaTeX Template 
% Version 1.3 (29/10/16)
%
% This template has been downloaded from:
% http://www.LaTeXTemplates.com
%
% Original author:
% Xavier Danaux (xdanaux@gmail.com) with modifications by:
% Vel (vel@latextemplates.com)
%
% License:
% CC BY-NC-SA 3.0 (http://creativecommons.org/licenses/by-nc-sa/3.0/)
%
% Important note:
% This template requires the moderncv.cls and .sty files to be in the same 
% directory as this .tex file. These files provide the resume style and themes 
% used for structuring the document.
%
%%%%%%%%%%%%%%%%%%%%%%%%%%%%%%%%%%%%%%%%%

%----------------------------------------------------------------------------------------
%	PACKAGES AND OTHER DOCUMENT CONFIGURATIONS
%----------------------------------------------------------------------------------------

\documentclass[11pt,a4paper,sans]{moderncv} % Font sizes: 10, 11, or 12; paper sizes: a4paper, letterpaper, a5paper, legalpaper, executivepaper or landscape; font families: sans or roman

\moderncvtheme[blue]{classic} 

%\moderncvstyle{casual} % CV theme - options include: 'casual' (default), 'classic', 'oldstyle' and 'banking'
%\moderncvcolor{blue} % CV color - options include: 'blue' (default), 'orange', 'green', 'red', 'purple', 'grey' and 'black'

\usepackage{url}
	
\usepackage{lipsum} % Used for inserting dummy 'Lorem ipsum' text into the template

\usepackage{xcolor}
\usepackage{graphicx}
\usepackage{fancyvrb}
\usepackage{wasysym}
\newcommand{\Rlogo}{\protect\includegraphics[height=2ex,keepaspectratio]{pictures/Rlogo-1.png}}
\newcommand{\MATLAB}{\protect\includegraphics[height=2ex,keepaspectratio]{pictures/Matlab_Logo.png}}
\newcommand{\MySQL}{\protect\includegraphics[height=2ex,keepaspectratio]{pictures/1200px-MySQL.svg.png}}
\newcommand{\QGIS}{\protect\includegraphics[height=2ex,keepaspectratio]{pictures/QGis_Logo.png}}
\newcommand{\ArcGIS}{\protect\includegraphics[height=2ex,keepaspectratio]{pictures/ArcGIS_logo.png}}
\newcommand{\Julia}{\protect\includegraphics[height=2ex,keepaspectratio]{pictures/julia_logo.PNG}}
\newcommand{\Shell}{\protect\includegraphics[height=2ex,keepaspectratio]{pictures/shell.jpg}}
\newcommand{\Git}{\protect\includegraphics[height=2ex,keepaspectratio]{pictures/git_logo.png}}
\newcommand{\Css}{\protect\includegraphics[height=2ex,keepaspectratio]{pictures/css.png}}
\usepackage{fontawesome5}

\usepackage[scale=0.95, top = 1cm, nofoot]{geometry} % Reduce document margins
\setlength{\hintscolumnwidth}{2.5cm} % Uncomment to change the width of the dates column
% \setlength{\makecvtitlenamewidth}{10cm} % For the 'classic' style, uncomment to adjust the width of the space allocated to your name

%----------------------------------------------------------------------------------------
%	NAME AND CONTACT INFORMATION SECTION
%----------------------------------------------------------------------------------------
% \vspace{-50cm}
\firstname{François} % Your first name
\familyname{Leroy} % Your last name
% All information in this block is optional, comment out any lines you don't need
\title{Data Science, Machine Learning, Deep Learning, Statistical Modelling, Geospatial analyses, Remote sensing}
\address{466 W 2nd Av.}{Columbus, OH 43201, USA}
\mobile{+420732642193}
%\phone{+420 737 480 623}
% \fax{(000) 111 1113}
\email{francois.libert.leroy@gmail.com}
\homepage{github.com/FrsLry}{https://github.com/FrsLry} % The first argument is the url for the clickable link, the second argument is the url displayed in the template - this allows special characters to be displayed such as the tilde in this example
% \homepage{www:\href{https://frslry.github.io/}{https://frslry.github.io/} \\ \href{https://github.com/FrsLry}{GitHub} (FrsLry)}
% \photo[90pt][0.4pt]{pictures/photo_CV.jpg} % The first bracket is the picture height, the second is the thickness of the frame around the picture (0pt for no frame)
% \quote{"A witty and playful quotation" - John Smith}


%----------------------------------------------------------------------------------------

\begin{document}

%----------------------------------------------------------------------------------------
%	COVER LETTER
%----------------------------------------------------------------------------------------

% To remove the cover letter, comment out this entire block

% \clearpage

% \recipient{Department of Applied Geography and Spatial Planning}{Czech University of Life Sciences Prague\\Kamýcká 129\\165 21 Praha 6 - Suchdol} % Letter recipient
% \date{\today} % Letter date
% \opening{Dear Dr. Keil,} % Opening greeting
% \closing{Sincerely,} % Closing phrase
% % \enclosure[Attached]{curriculum vit\ae{}} % List of enclosed documents

% \makelettertitle % Print letter title

% Assessing global-scale patterns of biodiversity has always been challenging and is, in this current context of anthropogenic threats, a question to face.  Scientific literature abounds in study cases and hypotheses regarding the main drivers of biodiversity, yet still no consensus has been reached.  Some of the most important limitations are the quantity and type of biodiversity data presently available, which are either incomplette or non-integrable (e.g. structured and unstructured data). Moreover, the effects of biodiversity drivers are dependent on the scale at which we observe the ecosystem. Combining these types of data (local/global and structured/unstructured) is essential for assessing species potential and realized niches. However, these datasets should not all be considered with the same strengths and bias need to be established.\newline\newline
% Throughout my MSc “Marine Sciences” at Sorbonne University and my internships, I have acquired a strong conceptual basis, especially in biostatistics, biogeography, marine ecology and oceanography, which have enabled me to consider ecosystems as a whole as a result of biotic and abiotic interactions. I have learnt to use many tools to study biodiversity, ecology and environmental conditions. Mapping with GIS softwares (ArcGIS, QGIS) has been part of my degree and of one of my internships. Moreover, I am comfortable using programming languages such as R and MATLAB to make statistical inferences from databases and to train ecosystem models. These two latter approaches have been at the centre of my two MSc internships. Indeed, statistics and modelling are the two numerical tools I have focused on. My first internship was about spatio-temporal variations in the recruitment age of an amphidromous species from the Indian Ocean using statistical tests and larval dispersion modelling. I am currently doing my final MSc internship on modelling the responses of the community associated to a habitat-forming species, the honeycomb worm reef (Sabellaria alveolata). This internship is enabling me to become familiar with in silico studies through using programming languages (mainly R).\newline\newline
% Using these knowledge, biodiversity data and modelling skills, one of the main steps of this PhD would be for me to shape a model that answers the following question: how can we assess species distributions and biodiversity drivers using cross-scale and heterogeneous biotic and abiotic data? After integrating different datasets from diverse sources, statistical inferences of species distributions would be the logical sequel. Particular attention will have to be paid on the origin and quality of those databases, especially the ones that may include bias. Given the fact that we will have a significant amount of data and not much prior knowledge, non-parametric machine learning may be another way to train a model. Ultimately, this model will be used in order to create GIS maps of predicted patterns and temporal evolution of biodiversity that take in account both environmental (i.e. local) and biogeographic (i.e. regional) effects. All of this in silico work could be shared on an online platform in order to present and discuss with the scientific community.\newline\newline 
% I would welcome the opportunity to talk with you more about the position. Thank you for your consideration, and do not hesitate to contact me if you need additional information. 

% \makeletterclosing % Print letter signature

% \newpage

%----------------------------------------------------------------------------------------
%	CURRICULUM VITAE
%----------------------------------------------------------------------------------------

\makecvtitle % Print the CV title

\vspace{-1.4cm}

%----------------------------------------------------------------------------------------
%	EDUCATION SECTION
%----------------------------------------------------------------------------------------
%\vspace{-1.6cm}
\section{Experience}

\cventry{2024-Ongoing\\~\\\includegraphics[width=8mm]{pictures/ohio_logo.JPG}}{Postdoctoral researcher - Using Artifical Inteligence to understand spatio-temporal changes of biodiversity}{\href{https://www.osu.edu/}{The Ohio State University}, dept. of Evolution, Ecology and Organismal Biology}{Columbus, Ohio}{}{
\begin{itemize}
    \item Creating Hierarchical Neural Networks from a Bayesian framework  
    \item Tailoring loss functions to infer distribution parameters
    \item Creating simulated data to test the model
    \item Hired for the project \href{https://www.abcresearchcenter.org/}{ABC Global Center}, collaborating with computer scientists from MIT, McGill University, Mila institute (Montreal, Canada) 
    \newline
\end{itemize}
}

\cventry{2020-2024\\\includegraphics[width=18mm]{pictures/CZU_logo_cerna.png}}{PhD. - Spatial scaling and decomposition of macroecological changes}{\href{https://www.fzp.czu.cz/en/}{Faculty of Environmental Sciences}, CZU, dept. of \href{https://www.fzp.czu.cz/en/r-9407-departments/r-9471-departments/r-9649-department-of-applied-geoinformatics-and-spatial-planning}{Spatial sciences}}{Prague}{}{
\begin{itemize}
    \item Using big data to model biodiversity changes across spatial scales
    \item Using model optimisation and selection on various machine learning algorithms (Random Forest, BRT, XGBoost, linear models...)
    \item Available \href{https://theses.cz/id/aigse5/}{here}
    \newline
\end{itemize}
}

\section{Projects}

\cventry{2020-2025}{}{}{}{}{
    \begin{itemize}
        \item Hierarchical models, mixed models using Bayesian inference for time-series analyses, Generalized Additive Models, Hidden Markov Models (\href{https://doi.org/10.32942/X21032}{article})
        \item Comparing model performance of Random Forest, Boosted Regression Trees, XGBoost and GLM using repeated cross-validation (\href{https://nsojournals.onlinelibrary.wiley.com/doi/10.1111/ecog.06995}{article})
        \item Feature importance and partial dependence from CART-based algorithm to explain measurement error of NASA's ICESat-2 (\href{https://doi.org/10.1016/j.rse.2022.113112}{article})
        \item Full publication list: \href{https://scholar.google.com/citations?user=t_chaRYAAAAJ&hl=en}{here}
    \end{itemize} 
}
   

\section{Teaching}

\cventry{2021-2024}{}{}{}{}{
\begin{itemize}
    \item Statistical ecology and macroecology
    \item Version Control using Git and Github
    \item GIS and spatial analysis
\end{itemize}    
}


\section{Education}

\cventry{2021\\(1 semester)}{Machine Learning}{Faculty of Mathematics and Physics, UFAL, Charles Univesity}{}{Prague}{Studying all Machine Learning algorithms, from Support Vector Machines to Neural Networks}{}{}

\cventry{2020-2024}{PhD. - Spatial scaling and decomposition of macroecological changes}{}{}{Prague}{}{}

\cventry{2018-2020\\\includegraphics[width=18mm]{pictures/Sciences_SU.png}}{Marine Sciences MSc}{Sorbonne University}{Paris}{}{Numerical Ecology, modelling, geostatistics, GIS,  oceanography, marine ecology,  biogeochemistry, database management \newline}

% Arguments not required can be left empty

\cventry{2015-2018}{Bachelor of Science}{South Brittany University}{Vannes (France)}{}{Specialized in Coastal Ecosystems and Management, GIS}

% %----------------------------------------------------------------------------------------
% %	COMPUTER SKILLS SECTION
% %----------------------------------------------------------------------------------------
% \vspace*{37px}
\section{Modelling skills}

\cvitem{\textbf{Deep Learning}}{Multilayer Perceptron, Convolutional/Recurrent/Hierarchical Neural Networks}
\cvitem{\textbf{Machine Learning}}{Classification and Regression Tree based algorithms (RF, BRT, GBM, XGBoost), Support Vector Machines, K-Nearest Neighbors, Naive Bayes, Linear Models (GLM, Mixed Models, polynomial regressions...), Hierarchical modelling}
\cvitem{\textbf{Others}}{Generalized Additive Models, Bayesian inference with MCMC algorithms, Bayesian Networks, Hidden Markov Models, Feature engineering, Spatially explicit models, Time series analysis, Multivariate analysis, Multiscale analysis, Clustering, Ordination, Model: optimisation (e.g. regularization), prediction, scalability} 

% %----------------------------------------------------------------------------------------
% %	COMPUTER SKILLS SECTION
% %----------------------------------------------------------------------------------------
\vspace*{35px}
\section{Programming skills}

\cvitem{Advanced}{\faPython Python, \Rlogo, \Git Git, \QGIS QGIS, \ArcGIS ArcGIS, \LaTeX}
% \faLinux Linux
\cvitem{Intermediate}{\Shell Shell, \MySQL MySQL}% \faAdobe Adobe Creative Cloud, Agisoft Metashape}
\cvitem{Basic}{\Julia Julia,\MATLAB MATLAB,\faHtml5 HTML5, \Css CSS}


\section{Internships and others}

\cventry{2024\\(1 semester)}{Deep Learning}{Faculty of Mathematics and Physics}{Charles University, Prague}{}{
\begin{itemize}
    \item Going through all Deep Learning algorithms
\end{itemize}
}


% \cventry{2023\\(1 week)}{\href{https://theodatasci.github.io/}{TheoMoDiv} workshop}{CESAB}{Montpellier}{}{
% \begin{itemize}
%     \item Training in theory-based approaches to model ecological data (time series, macroecology, interaction, trophic network)
% \end{itemize}
% }

% \cventry{2022\\(1 months)}{Visiting Ohio State University}{\href{http://jarzynalab.com/}{Jarzyna lab}}{Colombus, Ohio}{}{
% \begin{itemize}
%     \item Collaborating with Dr. Marta Jarzyna on the spatial scaling of abundance-based biodiversity trends
% \end{itemize}
% }

\cventry{2022\\(1 week)}{HMSC}{Jyväskylä summer school}{Jyväskylä , Finland}{}{
\begin{itemize}
    \item Summer school on Hirearchical Modeling of Species Community
\end{itemize}
}

\cventry{2020\\(6 months)}{Community Modelling}{\href{https://wwz.ifremer.fr/dyneco/Lab.-Lebco}{DYNECO-LEBCO}, IFREMER}{Brest (France)}{}{
\begin{itemize} 
\item Bayesian networks and qualitative modelling to assess the impact of environmental changes on the benthic communities
\end{itemize}
}

\cventry{2019\\(2 months)}{Numerical Ecology}{\href{https://borea.mnhn.fr/}{UMR BOREA} - \href{https://www.mnhn.fr/}{MNHN} - \href{https://www.locean-ipsl.upmc.fr/index.php?lang=fr}{LOCEAN}}{Paris (France)}{}{
\begin{itemize}
\item Ordinations, ANOVA, and Lagrangian ocean analysis
\end{itemize}
}

% \cventry{2018\\(2 months)}{Geospatial Study}{Géoarchitecture Laboratory}{Vannes (France)}{}{
% \begin{itemize}
% \item \textbf{Objective:} Use the opportunistic behavior of the European shag as a proxy to assess underwater biodiversity
% \end{itemize}
% }

\cventry{2017\\(5 months)}{Cartography, Photogrammetry}{Geosciences Ocean Laboratory}{Vannes (France)}{}{
\begin{itemize}
\item Study coastal dynamics by production of DEMs (Digital Elevation Models) for GIS analysis
\end{itemize}
}


%----------------------------------------------------------------------------------------
%	WORK EXPERIENCE SECTION
%----------------------------------------------------------------------------------------

% \section{Experience}

% \subsection{Vocational}

% \cventry{2012--Present}{1\textsuperscript{st} Year Analyst}{\textsc{Lehman Brothers}}{Los Angeles}{}{Developed spreadsheets for risk analysis on exotic derivatives on a wide array of commodities (ags, oils, precious and base metals), managed blotter and secondary trades on structured notes, liaised with Middle Office, Sales and Structuring for bookkeeping.
% \newline{}\newline{}
% Detailed achievements:
% \begin{itemize}
% \item Learned how to make amazing coffee
% \item Finally determined the reason for \textsc{PC LOAD LETTER}:
% \begin{itemize}
% \item Paper jam
% \item Software issues:
% \begin{itemize}
% \item Word not sending the correct data to printer
% \item Windows trying to print in letter format
% \end{itemize}
% \item Coffee spilled inside printer
% \end{itemize}
% \item Broke the office record for number of kitten pictures in cubicle
% \end{itemize}}

% %------------------------------------------------

% \cventry{2011--2012}{Summer Intern}{\textsc{Lehman Brothers}}{Los Angeles}{}{Rated "truly distinctive" for Analytical Skills and Teamwork.}

% %------------------------------------------------

% \subsection{Miscellaneous}

% \cventry{2010--2011}{}{}{}{}{Spent some time finding myself. This was a courageous endeavour that didn't have a job title. It was quite important to my overall development though so I'm adding it to my CV. Also it explains the gap in my otherwise stellar CV.}

% \cventry{2009--2010}{Computer Repair Specialist}{Buy More}{Burbank}{}{Worked in the Nerd Herd and helped to solve computer problems. Allowed me to become expert in all forms of martial arts and weaponry.}


%\cvitem{}{\underline{Leroy, F.}, Reif, J., Storch, D., \& Keil, P. (2022). How has bird biodiversity changed over time? A review across spatio-temporal scales. \textbf{\textit{EcoEvoRxiv}}(preprint). \href{https://doi.org/10.32942/osf.io/jhr6v}{https://doi.org/10.32942/osf.io/jhr6v}}
% \cvitem{2010}{Top Achiever Award -- Commerce}


% %----------------------------------------------------------------------------------------
% %	Teaching SECTION
% %----------------------------------------------------------------------------------------

% \cvitem{2009}{Poster at the Annual Business Conference in Oregon}

% %----------------------------------------------------------------------------------------
% %	LANGUAGES SECTION
% %----------------------------------------------------------------------------------------

%\section{Languages}

%\cventry{}{}{}{}{}{French (mothertongue), English (fluent speaking, reading, writing), Spanish (basic)}
% \cventry{}{}{}{}{}{}{}{}

% \cvitemwithcomment{French}{Mothertongue}{}
% \cvitemwithcomment{English}{Fluent}{Speaking/Writing/Reading}
% \cvitemwithcomment{Spanish}{Basic}{}

% %----------------------------------------------------------------------------------------
% %	Talks SECTION
% %----------------------------------------------------------------------------------------

% \vspace*{29px}
\section{HIghlighted Talks and Conferences}

% \cventry{Conference\\2024-06-14}{\textbf{Acceleration and demographic rates of bird abundance decline in North America}}{GfO macro}{Marburg, Germany}{\href{https://frslry.github.io/Gfo_macro/}{Slides}}{%\underline{Content:}
% \begin{itemize}
% \item Spatial scaling of species richness trends
% \item Birds of the Czech Republic
% \item Positive and stronger trend of species richness with increasing spatial scale 
% \item Explained by spatial scaling of colonization, extinction and persistence
% \newline
% \end{itemize}
% }

\cventry{Invited speaker\\2024-02-16}{\textbf{Introduction to Reproducible Science: Version Control using Git and Github}}{Ecoinformatics IAVS}{Online}{\href{https://frslry.github.io/git_pres/}{Slides}}{%\underline{Content:}
% \begin{itemize}
% \item Spatial scaling of species richness trends
% \item Birds of the Czech Republic
% \item Positive and stronger trend of species richness with increasing spatial scale 
% \item Explained by spatial scaling of colonization, extinction and persistence
% \newline
% \end{itemize}
}

% \cventry{Conference\\2024-01-07}{\textbf{Acceleration and demographic rates of bird decline in North America}}{International Biogeography Society}{Prague}{\href{https://github.com/FrsLry/IBS_Prague_2024/blob/main/poster_IBS_Prague_2024.jpg}{Poster}}{%\underline{Content:}
% \begin{itemize}
% \item Spatial scaling of species richness trends
% \item Birds of the Czech Republic
% \item Positive and stronger trend of species richness with increasing spatial scale 
% \item Explained by spatial scaling of colonization, extinction and persistence
% \newline
% \end{itemize}
% }

\cventry{Conference\\2023-08-10}{\textbf{Decomposing abundance change to recruitment and loss: analysis of the North-American avifauna}}{Ecological Society of America}{Portland, OR}{\href{https://frslry.github.io/ESA_conf/}{Slides}}{%\underline{Content:}
% \begin{itemize}
% \item Spatial scaling of species richness trends
% \item Birds of the Czech Republic
% \item Positive and stronger trend of species richness with increasing spatial scale 
% \item Explained by spatial scaling of colonization, extinction and persistence
% \newline
% \end{itemize}
}

\cventry{Conference\\2022-06-05}{\textbf{Untangling biodiversity changes across a continuum of spatial scales}}{International Biogeography Society conference}{Vancouver, BC}{\href{https://frslry.github.io/IBS_conf/}{Slides}}{%\underline{Content:}
% \begin{itemize}
% \item Spatial scaling of species richness trends
% \item Birds of the Czech Republic
% \item Positive and stronger trend of species richness with increasing spatial scale 
% \item Explained by spatial scaling of colonization, extinction and persistence
% \newline
% \end{itemize}
}

\cventry{Conference\\2021-10-23}{\textbf{Modeling biodiversity changes across a continuum of spatial scales}}{International Biogeography Society conference (Early career)}{Online}{\href{https://frslry.github.io/Gfo_pres/}{Slides}}{%\underline{Content:}
% \begin{itemize}
% \item Using machine learning methods to model species richness trends across spatial scales
% \item Using models output to highlight the influence of spatio-temporal grains
% \item Taxon: birds
% \item Study extent: Czech Republic
% \newline
% \end{itemize}
}

% \cventry{Conference\\2021-09-01}{\textbf{Spatio-temporal scaling of biodiversity trends}}{\href{https://www.gfoe-conference.de/}{GfÖ} Virtual Annual Meeting}{Online}{\href{https://frslry.github.io/Gfo_pres/}{Slides}}{%\underline{Content:}
% \begin{itemize}
% \item Pilot results of my PhD
% \item Highlighting the spatial scaling of biodiversity trends 
% \item Taxon: birds
% \item Study extent: Czech Republic
% \newline
% \end{itemize}
% }

% \cventry{Seminar\\2020-07-01}{\textbf{Introduction to Reproducible Science: Version Control using Git}}{CZU}{Prague}{\href{https://frslry.github.io/git_pres/}{Slides}}{%\underline{Content:}
% \begin{itemize}
% \item Why is reproducible science essential?
% \item What is a version control software?
% \item How to use git and github from the command line?
% \item How to share your work with Github?
% \newline
% \end{itemize}
% }

% %----------------------------------------------------------------------------------------
% %	Publication SECTION
% %----------------------------------------------------------------------------------------

% \nocite{*}
% \bibliographystyle{plainurl}
% \bibliography{references.bib}

% %----------------------------------------------------------------------------------------
% %	Referees SECTION
% %----------------------------------------------------------------------------------------
% \vspace*{37px}
\section{Scientific Referees}

\cvitem{}{Dr. \textbf{Marta Jarzyna}, Ohio State University, \phone{+1 (978) 587-5938}, \href{jarzyna.1@osu.edu}{jarzyna.1@osu.edu}}
\cvitem{}{Dr. \textbf{Petr Keil}, Czech University of Life Sicences, \phone{+420 224382659}, \href{keil@fzp.czu.cz}{keil@fzp.czu.cz}}
\cvitem{}{Dr. \textbf{Martin Marzloff}, Ifremer, \phone{+332 98224327}, \href{Martin.Marzloff@ifremer.fr}{Martin.Marzloff@ifremer.fr}}
\cvitem{}{Dr. \textbf{Vítězslav Moudrý}, Czech University of Life Sicences, \phone{+420 224382653}, \href{moudry@fzp.czu.cz}{moudry@fzp.czu.cz}}

\end{document}