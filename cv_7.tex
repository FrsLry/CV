%%%%%%%%%%%%%%%%%%%%%%%%%%%%%%%%%%%%%%%%%
% "ModernCV" CV and Cover Letter
% LaTeX Template
% Version 1.3 (29/10/16)
%
% This template has been downloaded from:
% http://www.LaTeXTemplates.com
%
% Original author:
% Xavier Danaux (xdanaux@gmail.com) with modifications by:
% Vel (vel@latextemplates.com)
%
% License:
% CC BY-NC-SA 3.0 (http://creativecommons.org/licenses/by-nc-sa/3.0/)
%
% Important note:
% This template requires the moderncv.cls and .sty files to be in the same 
% directory as this .tex file. These files provide the resume style and themes 
% used for structuring the document.
%
%%%%%%%%%%%%%%%%%%%%%%%%%%%%%%%%%%%%%%%%%

%----------------------------------------------------------------------------------------
%	PACKAGES AND OTHER DOCUMENT CONFIGURATIONS
%----------------------------------------------------------------------------------------

\documentclass[11pt,a4paper,sans]{moderncv} % Font sizes: 10, 11, or 12; paper sizes: a4paper, letterpaper, a5paper, legalpaper, executivepaper or landscape; font families: sans or roman

\moderncvtheme[blue]{classic} 

%\moderncvstyle{casual} % CV theme - options include: 'casual' (default), 'classic', 'oldstyle' and 'banking'
%\moderncvcolor{blue} % CV color - options include: 'blue' (default), 'orange', 'green', 'red', 'purple', 'grey' and 'black'

\usepackage{url}
	
\usepackage{lipsum} % Used for inserting dummy 'Lorem ipsum' text into the template

\usepackage{xcolor}
\usepackage{graphicx}
\usepackage{fancyvrb}
\newcommand{\Rlogo}{\protect\includegraphics[height=2ex,keepaspectratio]{pictures/Rlogo-1.png}}
\newcommand{\MATLAB}{\protect\includegraphics[height=2ex,keepaspectratio]{pictures/Matlab_Logo.png}}
\newcommand{\MySQL}{\protect\includegraphics[height=2ex,keepaspectratio]{pictures/1200px-MySQL.svg.png}}
\newcommand{\QGIS}{\protect\includegraphics[height=2ex,keepaspectratio]{pictures/QGis_Logo.png}}
\newcommand{\ArcGIS}{\protect\includegraphics[height=2ex,keepaspectratio]{pictures/ArcGIS_logo.png}}
\newcommand{\Julia}{\protect\includegraphics[height=2ex,keepaspectratio]{pictures/julia_logo.PNG}}
\newcommand{\Shell}{\protect\includegraphics[height=2ex,keepaspectratio]{pictures/shell.jpg}}
\newcommand{\Git}{\protect\includegraphics[height=2ex,keepaspectratio]{pictures/git_logo.png}}
\newcommand{\Css}{\protect\includegraphics[height=2ex,keepaspectratio]{pictures/css.png}}
\usepackage{fontawesome5}

\usepackage[scale=0.95, top = 0.1cm, nofoot]{geometry} % Reduce document margins
\setlength{\hintscolumnwidth}{2.5cm} % Uncomment to change the width of the dates column
% \setlength{\makecvtitlenamewidth}{10cm} % For the 'classic' style, uncomment to adjust the width of the space allocated to your name

%----------------------------------------------------------------------------------------
%	NAME AND CONTACT INFORMATION SECTION
%----------------------------------------------------------------------------------------
% \vspace{-5cm}
\firstname{François} % Your first name
\familyname{Leroy} % Your last name
% All information in this block is optional, comment out any lines you don't need
\title{Keywords: Numerical Ecology, Modeling, Machine Learning}
\address{10 Rue des Préaux, Saint-Marcel}{27950, Normandy, France}
\mobile{+420737480623}
%\phone{+420 737 480 623}
% \fax{(000) 111 1113}
\email{leroy@fzp.czu.cz}
% \homepage{staff.org.edu/~jsmith}{staff.org.edu/$\sim$jsmith} % The first argument is the url for the clickable link, the second argument is the url displayed in the template - this allows special characters to be displayed such as the tilde in this example
\extrainfo{\href{https://frslry.github.io/}{Website} \\ \href{https://github.com/FrsLry}{GitHub} (FrsLry)}
\photo[90pt][0.4pt]{pictures/photo_CV.jpg} % The first bracket is the picture height, the second is the thickness of the frame around the picture (0pt for no frame)
% \quote{"A witty and playful quotation" - John Smith}


%----------------------------------------------------------------------------------------

\begin{document}

%----------------------------------------------------------------------------------------
%	COVER LETTER
%----------------------------------------------------------------------------------------

% To remove the cover letter, comment out this entire block

% \clearpage

% \recipient{Department of Applied Geography and Spatial Planning}{Czech University of Life Sciences Prague\\Kamýcká 129\\165 21 Praha 6 - Suchdol} % Letter recipient
% \date{\today} % Letter date
% \opening{Dear Dr. Keil,} % Opening greeting
% \closing{Sincerely,} % Closing phrase
% % \enclosure[Attached]{curriculum vit\ae{}} % List of enclosed documents

% \makelettertitle % Print letter title

% Assessing global-scale patterns of biodiversity has always been challenging and is, in this current context of anthropogenic threats, a question to face.  Scientific literature abounds in study cases and hypotheses regarding the main drivers of biodiversity, yet still no consensus has been reached.  Some of the most important limitations are the quantity and type of biodiversity data presently available, which are either incomplette or non-integrable (e.g. structured and unstructured data). Moreover, the effects of biodiversity drivers are dependent on the scale at which we observe the ecosystem. Combining these types of data (local/global and structured/unstructured) is essential for assessing species potential and realized niches. However, these datasets should not all be considered with the same strengths and bias need to be established.\newline\newline
% Throughout my MSc “Marine Sciences” at Sorbonne University and my internships, I have acquired a strong conceptual basis, especially in biostatistics, biogeography, marine ecology and oceanography, which have enabled me to consider ecosystems as a whole as a result of biotic and abiotic interactions. I have learnt to use many tools to study biodiversity, ecology and environmental conditions. Mapping with GIS softwares (ArcGIS, QGIS) has been part of my degree and of one of my internships. Moreover, I am comfortable using programming languages such as R and MATLAB to make statistical inferences from databases and to train ecosystem models. These two latter approaches have been at the centre of my two MSc internships. Indeed, statistics and modelling are the two numerical tools I have focused on. My first internship was about spatio-temporal variations in the recruitment age of an amphidromous species from the Indian Ocean using statistical tests and larval dispersion modelling. I am currently doing my final MSc internship on modelling the responses of the community associated to a habitat-forming species, the honeycomb worm reef (Sabellaria alveolata). This internship is enabling me to become familiar with in silico studies through using programming languages (mainly R).\newline\newline
% Using these knowledge, biodiversity data and modelling skills, one of the main steps of this PhD would be for me to shape a model that answers the following question: how can we assess species distributions and biodiversity drivers using cross-scale and heterogeneous biotic and abiotic data? After integrating different datasets from diverse sources, statistical inferences of species distributions would be the logical sequel. Particular attention will have to be paid on the origin and quality of those databases, especially the ones that may include bias. Given the fact that we will have a significant amount of data and not much prior knowledge, non-parametric machine learning may be another way to train a model. Ultimately, this model will be used in order to create GIS maps of predicted patterns and temporal evolution of biodiversity that take in account both environmental (i.e. local) and biogeographic (i.e. regional) effects. All of this in silico work could be shared on an online platform in order to present and discuss with the scientific community.\newline\newline 
% I would welcome the opportunity to talk with you more about the position. Thank you for your consideration, and do not hesitate to contact me if you need additional information. 

% \makeletterclosing % Print letter signature

% \newpage

%----------------------------------------------------------------------------------------
%	CURRICULUM VITAE
%----------------------------------------------------------------------------------------

\makecvtitle % Print the CV title

\vspace{-1.4cm}


%----------------------------------------------------------------------------------------
%	EDUCATION SECTION
%----------------------------------------------------------------------------------------
%\vspace{-1.6cm}
\section{Experience}

\cventry{2020-2024\\Ongoing\\\includegraphics[width=30mm]{pictures/CZU_logo_cerna.png}}{PhD. Modeling spatio-temporal biodiversity changes across scales}{\href{https://www.fzp.czu.cz/en/}{Faculty of Environmental Sciences}, CZU, dept. of \href{https://www.fzp.czu.cz/en/r-9407-departments/r-9471-departments/r-9649-department-of-applied-geoinformatics-and-spatial-planning}{Spatial sciences}}{Prague}{}{
\begin{itemize}
    \item Modeling biodiversity using machine learning, frequentists and bayesian methods
    \item Programming
    \item English communication skills (both oral and writing)
    \item Supervised by \href{https://petrkeil.github.io/website/}{Dr. Petr Keil}
    \newline
\end{itemize}
}

\cventry{2018--2020\\\includegraphics[width=18mm]{pictures/Sciences_SU.png}}{\href{http://sciencesdelamer.sorbonne-universite.fr/}{Marine Sciences} MSc}{Sorbonne University}{Paris (France, graduated September 2020)}{}{\textcolor{red}{Numerical Ecology, modelling, geostatistics, GIS,}  oceanography, marine \textcolor{red}{ecology,}  biogeochemistry, database management \newline}

% Arguments not required can be left empty

\cventry{2017--2018}{3\textsuperscript{rd} year of Bachelor of Science}{South Brittany University}{Vannes (France)}{}{Specialized in Coastal Ecosystems and Management, \textcolor{red}{GIS} \newline}

\cventry{2015--2017}{1\textsuperscript{st} and 2\textsuperscript{nd} year of Bachelor of Science}{Rouen Normandy University}{Rouen (France)}{}{Specialized in Botanic}


\section{Internships}

\cventry{2020\\(6 months)\\\includegraphics[width=20mm]{pictures/Ifremer.png}}{Community \textcolor{red}{modelling} }{\href{https://wwz.ifremer.fr/dyneco/Lab.-Lebco}{DYNECO-LEBCO}, IFREMER}{Brest (France)}{}{
\begin{itemize} 
\item \textbf{Objective:} develop a simulation tool to assess dynamic communities accompanying biogenic reefs built by \textit{Sabellaria alveolata} (Linnaeus,
1767)(honeycomb worm) 
\item Explore the community topology using \textcolor{red}{qualitative modelling} (Dambacher \textit{et al.} 2002, Marzloff \textit{et al.} 2016)
\item Infer a \textcolor{red}{Dynamic Bayesian Network}  (BN) from a large database (\href{http://www.honeycombworms.org/The-REEHAB-Project}{REEHAB project})
%\item Develop a \textcolor{red}{Dynamic Bayesian Network}  (DBN) of the %community \newline
\end{itemize}
}


\cventry{2019\\(2 months)\\\includegraphics[width=15mm]{pictures/MNHN-logo.jpg}}{\textcolor{red}{Numerical ecology}  study}{\href{https://borea.mnhn.fr/}{UMR BOREA} - \href{https://www.mnhn.fr/}{MNHN} - \href{https://www.locean-ipsl.upmc.fr/index.php?lang=fr}{LOCEAN} }{Paris (France)}{}{
\begin{itemize}
\item \textbf{Objective:} spatiotemporal recruitement variability of \textit{Sicyopterus lagocephalus} (Pallas 1770)(Teleostei : Gobiidae : Sicydiinae), amphidromous species of the Indian Ocean   
\item Pelagic Larval Duration (PLD) determination by otolithometry
\item \textcolor{red}{Statistical analysis}  to observe spatial (rivers) and temporal (season/year) differences of those PLD
\item Larval dispersion \textcolor{red}{modelling} using the Ichthyop lagrangian model in backward to assess larval provenance\newline
\end{itemize}
}

\cventry{2018\\(2 months)}{Ecological study}{Géoarchitecure Laboratory}{Vannes (France)}{}{
\begin{itemize}
\item \textbf{Objective:} use the opportunistic feature of the European shag to assess fish biodiversity
%\item Rejection pellets dissection and harvesting
%\item Fish identification using otoliths, data analysis\newline
\end{itemize}
}

\cventry{2017\\(5 months)}{Mapping, Photogrammetry}{Géosciences Océans Laboratory}{Vannes (France)}{}{
\begin{itemize}
\item \textbf{Objective:} study the coastal dynamic of a beach in order to distribute sediment at the most relevant place
\item Three dimensional modelling of a beach to observe its evolution
\item Production of DEM (\textit{i.e.} Digital Elevation Model) to exploit in \textcolor{red}{GIS} software
\end{itemize}
}





%----------------------------------------------------------------------------------------
%	WORK EXPERIENCE SECTION
%----------------------------------------------------------------------------------------

% \section{Experience}

% \subsection{Vocational}

% \cventry{2012--Present}{1\textsuperscript{st} Year Analyst}{\textsc{Lehman Brothers}}{Los Angeles}{}{Developed spreadsheets for risk analysis on exotic derivatives on a wide array of commodities (ags, oils, precious and base metals), managed blotter and secondary trades on structured notes, liaised with Middle Office, Sales and Structuring for bookkeeping.
% \newline{}\newline{}
% Detailed achievements:
% \begin{itemize}
% \item Learned how to make amazing coffee
% \item Finally determined the reason for \textsc{PC LOAD LETTER}:
% \begin{itemize}
% \item Paper jam
% \item Software issues:
% \begin{itemize}
% \item Word not sending the correct data to printer
% \item Windows trying to print in letter format
% \end{itemize}
% \item Coffee spilled inside printer
% \end{itemize}
% \item Broke the office record for number of kitten pictures in cubicle
% \end{itemize}}

% %------------------------------------------------

% \cventry{2011--2012}{Summer Intern}{\textsc{Lehman Brothers}}{Los Angeles}{}{Rated "truly distinctive" for Analytical Skills and Teamwork.}

% %------------------------------------------------

% \subsection{Miscellaneous}

% \cventry{2010--2011}{}{}{}{}{Spent some time finding myself. This was a courageous endeavour that didn't have a job title. It was quite important to my overall development though so I'm adding it to my CV. Also it explains the gap in my otherwise stellar CV.}

% \cventry{2009--2010}{Computer Repair Specialist}{Buy More}{Burbank}{}{Worked in the Nerd Herd and helped to solve computer problems. Allowed me to become expert in all forms of martial arts and weaponry.}

% %----------------------------------------------------------------------------------------
% %	Publication SECTION
% %----------------------------------------------------------------------------------------

\nocite{*}
\bibliographystyle{plainurl}
\bibliography{references.bib}

%\cvitem{}{\underline{Leroy, F.}, Reif, J., Storch, D., \& Keil, P. (2022). How has bird biodiversity changed over time? A review across spatio-temporal scales. \textbf{\textit{EcoEvoRxiv}}(preprint). \href{https://doi.org/10.32942/osf.io/jhr6v}{https://doi.org/10.32942/osf.io/jhr6v}}
% \cvitem{2010}{Top Achiever Award -- Commerce}

% %----------------------------------------------------------------------------------------
% %	COMPUTER SKILLS SECTION
% %----------------------------------------------------------------------------------------

\section{Computer skills}

\cvitem{Basic}{\Julia Julia, \Shell Shell,\MATLAB MATLAB,\faHtml5 HTML5, \Css CSS}
% \faLinux Linux
\cvitem{Intermediate}{\faPython Python, \MySQL MySQL, \faAdobe Creative Cloud, Agisoft Metashape}
\cvitem{Advanced}{\Rlogo, \Git Git, \QGIS QGIS, \ArcGIS ArcGIS, \LaTeX}

% %----------------------------------------------------------------------------------------
% %	Teaching SECTION
% %----------------------------------------------------------------------------------------

\section{Teaching}

\cvitem{2021}{Introduction to GIS using ArcGIS \ArcGIS(14 hours)}
% \cvitem{2009}{Poster at the Annual Business Conference in Oregon}

% %----------------------------------------------------------------------------------------
% %	LANGUAGES SECTION
% %----------------------------------------------------------------------------------------

%\section{Languages}

%\cventry{}{}{}{}{}{French (mothertongue), English (fluent speaking, reading, writing), Spanish (basic)}
% \cventry{}{}{}{}{}{}{}{}

% \cvitemwithcomment{French}{Mothertongue}{}
% \cvitemwithcomment{English}{Fluent}{Speaking/Writing/Reading}
% \cvitemwithcomment{Spanish}{Basic}{}


% \section{Summer Jobs\newline}

% \cvlistdoubleitem{Hospital porter}{Waiter in a botanic café\newline}

% %----------------------------------------------------------------------------------------
% %	INTERESTS SECTION
% %----------------------------------------------------------------------------------------

% \section{Interests}

% % \renewcommand{\listitemsymbol}{-~} % Changes the symbol used for lists


% % \cvlistdoubleitem{Birdwatching}{Hiking}
% % \cvlistdoubleitem{Reading}{Running/Swimming}
% % \cvlistdoubleitem{Camping}{Travelling}
% % \cvlistdoubleitem{Playing drums}{Writing}

% \cventry{}{}{}{}{Birdwatching, hiking, coding, reading, running, swimminng, camping, travelling, playing drums, writing}{}{}

% \section{Scientific referees}
% \cventry{}{}{}{}{}{
% \begin{itemize}
% \item \href{https://annuaire.ifremer.fr/cv/17078/en/}{Dr. Stanislas % Dubois}, director of DYNECO-LEBCO Laboratory (IFREMER Brest, % France)\newline\emailsymbol\href{mailto:Stanislas.Dubois@ifremer.fr}{Stan% islas.Dubois@ifremer.fr} | \phonesymbol +33 (0)2 98 22 49 18
% \item \href{https://www.researchgate.net/profile/Celine_Ellien}{Dr. % Céline Ellien}, assistant professor at Sorbonne Université (SU), BOREA % Laboratory \newline\emailsymbol\href{mailto:celine.ellien@upmc.fr}{celine% .ellien@upmc.fr} | \phonesymbol +33 (0)1 40 79 57 48
% \end{itemize} }


\end{document}